\documentstyle[11pt,titlepage]{article}

\oddsidemargin 0.25in
\evensidemargin 0.25in

\textwidth 6.0in
\topmargin 0in
\textheight 8in

\itemsep 0pt
\marginparwidth 1in

% \psfigure{width in inches}{height in inches}{file}
% Modify the following definition to select figures on or off.
\newcommand{\psfigure}[3]{\psfigureon{#1}{#2}{#3}}
\newcommand{\psfigureoff}[3]{\setlength{\unitlength}{1in}\begin{picture}(#1,#2)\put(0,0){\framebox(#1,#2){#3}}\end{picture}}
\newcommand{\psfigureon}[3]{\setlength{\unitlength}{1in}\begin{picture}(#1,#2)\put(0,0){\special{psfile=#3}}\end{picture}}
\newcommand{\psfigureboxed}[3]{\setlength{\unitlength}{1in}\begin{picture}(#1,#2)\put(0,0){\special{psfile=#3}}\put(0,0){\framebox(#1,#2){}}\end{picture}}

\begin{document}

\begin{center}
\LARGE Authentication Service Design \\[0.25in]
\Large Robert Sansom \\
%\Large \today
\Large{August, 1988}
\vspace{0.25in}
\end{center}

\section{Overview}
The authentication service is used by servers to identify clients.  In
addition it provides servers with support for access control.  The
authentication service uses Mach ports as {\em tokens\/} of identity and
thus its security is dependent on the security of the underlying IPC
facility.

Before a user can obtain a token of identity, the user must log in to the
authentication service.  A user logs in by sending a user identifier and
password to the authentication service.  The authentication service checks
the password by encrypting it under a one-way cipher function and comparing
it with the stored version of the encrypted password.

If the log in is successful, then the authentication service creates a new
{\em private port\/} for the user.  The private port represents the user to
the authentication service and the authentication service treats any request
message received on the private port as already authenticated.  This private
port is useful only for communicating with the authentication service;  it
cannot be used for accessing resources guarded by other servers.  However
the user can send a request message to the private port asking the
authentication service to create a token of the user's identity.  The
authentication service creates the token and remembers that the new token
represents the user.

\begin{figure}[tb]
\begin{center}
\psfigure{5.5}{3.2}{"authexample.PS"}
\end{center}

\begin{small}
User A wishes to access resources managed by server B.  A obtains a token,
P1 from the authentication service.  A sends P1 to B, and B verifies P1 with
the AS.  If P1 is valid, and A is a member of a group that has access to
some subset of the resources that B manages, then B creates and returns a
new private port, P2, to A.  All reply messages in this protocol are sent to
the reply ports contained in the request messages.  All messages must be
sent securely if the authentication protocol is to be secure.
\end{small}
\caption{Summary of Authentication Process}
\label{auth-summary}
\end{figure}

As shown in Figure \ref{auth-summary}, once a user has obtained a token from
the authentication service, the user can pass the token to a server when
making a request of the server.  The server can {\em verify\/} the identity
of the user by querying the authentication service about the token.  If the
token is valid, the authentication service will return the identity of the
user that created the token.  In addition the authentication service will
return a list of the {\em access control groups\/} (see also Section
\ref{author}) represented by the token.  The list returned need not contain
all the access control groups of which the user is a member --- when asking
the authentication service to create a token, the user can specify that the
token should represent only a subset of these groups.

The server can then check the access control groups represented by the token
against the access control lists that it maintains to protect its resources.
If the user does have access to the requested resource, the server can
create and return to the user a new private port that the user can use to
access the resource.  The server will treat any requests that arrive on this
private port as already verified.

When a user logs out, either explicitly by using the authentication service
{\tt log\_out} call or implicitly by just exiting, the private port and all
the tokens associated with that user are deallocated.  Deleting the user's
tokens immediately upon the user's log out may cause problems if a server
has received a token but has not yet verified it with the authentication
service.  When the server comes to verify the token it will find that the
token is no longer valid and thus it will not be able to carry out the
user's original request.

\section{Implementation}
The authentication service is implemented by a combination of {\em Central
Authentication Servers\/} (CASs) and {\em Local Authentication Servers\/}
(LASs).  Ideally there should be multiple central authentication servers to
ensure the availability of the authentication service;  the current
implementation, however, has only a single central server.

\subsection{Central Authentication Server}
The central authentication server maintains an authentication database that
contains information about users and their memberships of access control
groups.  Currently this database is stored in the file named {\em
ASDATABASE}.  Passwords are stored in the authentication database under a
one-way encryption function as in the {\sc Unix}\footnote{{\sc Unix} is a
trademark of AT\&T Bell Laboratories.} password database.  Storing the
passwords in an encrypted form means that the authentication database does
not form a weak link in the security of the system.  Moreover, the database
can be made readable to all users thus allowing users easy access to the
public information about users' memberships of access control groups.

When a user logs in from a remote node, the unencrypted version of the
user's password must be received by the central authentication server.  The
central authentication server encrypts the password itself and compares it
against the stored, encrypted version to see if the log in request is valid.
The request message sent between the local and central authentication
servers must be sent securely in order to protect the security of a user's
passwords.

\subsection{Local Authentication Servers}
Each node in the distributed system, apart from the central node, has a
local authentication server that maintains a connection to a central
authentication server.  If the connection to the CAS fails, then the local
server will handle the failure gracefully.  In addition, the local
authentication server caches verification information obtained from
intercepting all authentication requests between tasks on its node and the
CAS.  Thus it will be able to answer many subsequent authentication requests
using its cached information.

The local authentication server creates and holds receive and ownership
rights to all ports resulting from authentication service requests.  Each
private port created as a result of a successful log in is actually a pair
of private ports, a local and central private port, both created by the
local authentication server.  Each token is a single port created by the
local authentication server.

When the user makes a log in request, the LAS creates a central private port
and passes send rights to this port to the CAS.  If the log in request is
successful then the LAS uses this port to represent the user in subsequent
authentication requests sent to the CAS.  But before replying to the user,
the LAS creates a local private port that it associates with the central
private port.  Send rights to this local private port are returned to the
user in the log in reply.

A user makes subsequent authentication requests by sending messages to the
local private port.  The local authentication server maps the local private
port into the corresponding central private port and forwards the request
message to the central authentication server.  Note that because the CAS
does not have receive rights to the central private port, the forwarded
request cannot be sent to the central private port but instead must be sent
to the CAS's public port.  For consistency the local request made by the
user is sent not to the local private port but to the local authentication
server's public port.

When a user makes a token creation request, the local authentication server
creates a port that will be the token.  It then requests that the central
authentication server {\em register\/} the port as a token.  Send rights to
the token are passed in the request sent to the CAS.  If the request is
successful, the LAS returns send rights for the token to the user task.

The local authentication server can detect that the user has logged out
without informing the authentication service if it receives a {\em no more
senders\/} notification\footnote{The {\em no more senders\/} notification
for a port implies that no tasks hold send rights to the port.  The current
implementation of Mach does not, however, support this notification.} for
the user's private port.  Similarly it can detect that a token is no longer
in use if it gets a {\em no more senders\/} notification for the token.

\subsection{Interface}
The following requests can be made of the authentication service (more
details can be found in the manual entries):
\begin{quotation} \noindent {\small
\verb"log-in(user-name,password)" $\rightarrow$ \verb"(private port)"}

Logs in a user if the clear-text password is correct for the given
user-name.  Returns send rights to a private port to which the user can send
subsequent requests to the authentication service.
\end{quotation}

\begin{quotation} \noindent {\small
\verb"log-out(private port)"}

Logs out the user associated with the given private port.
\end{quotation}

\begin{quotation} \noindent {\small
\verb"create-token(private port,group-ids)" $\rightarrow$ \verb"(token)"}

Creates and returns a token that represents the user associated with the
given private port.  By default the token will represent all the access
control groups of which the user is a member.  However an optional list of
access control groups identifiers can be passed in and, if these groups are
a subset of the groups of which the user is a member, the token will be
associated only with these access groups.  Thus the optional list will {\em
restrict\/} the protection domain denoted by the token.
\end{quotation}

\begin{quotation} \noindent {\small
\verb"delete-token(private port,token)"}

Deletes a token registered previously by the user who is represented by the
given private port.
\end{quotation}

\begin{quotation} \noindent {\small
\verb"verify-token-ids(token)" $\rightarrow$ \verb"(user-id,group-ids)"}

\noindent {\small
\verb"verify-token-ids(token)" $\rightarrow$ \verb"(user-name,group-names)"}

Returns the user identity and access control groups represented by a token.
Each access control group can be referred to either by its {\em identifier\/}
(a 32 bit unique integer) or by its {\em group-name\/} (an ASCII string). The
first form of the verify call returns access control group identifiers.  The
second form returns access control group names.
\end{quotation}

\begin{quotation} \noindent {\small
\verb"change-password(private-port,old-password,new-password)"}

Changes the password for a user.  Either the private port must be that of
the system administrator, who is allowed to change any user's password, or
the private port must represent the user requesting the password change.  In
the latter case the user's old password must also be correct.
\end{quotation}

\section{Access Control Support}
\label{author}
\subsection{Interface}
Servers that manage the resources of the distributed system must control
access to their resources and determine whether users are authorised to
access their resources.  In the secure Mach distributed system, servers use
{\em Access Control Lists\/} (ACLs) to make authorisation decisions.  An
access control list for a resource maps user identifiers on to specific
access rights to the resource.  Since individual servers know best what ACL
support they need, servers manage access control access control lists
themselves.

To aid servers in implementing access control lists, a simple scheme to
support access control is provided as part of the authentication service.
The scheme allows users to be members of {\em access control groups}.  An
access control group defines a subset of users to whom privileges may be
awarded or from whom privileges may be revoked.  With one verification call
a server can determine all (or a subset) of the access control groups to
which a user belongs.  The use of access control groups can simplify the
ACLs that the server needs to maintain at the cost of the management of the
groups.  However typical access control groups are relatively static
entities whose rate of change is slow.  For example they often reflect such
natural groupings as projects or courses.  Thus changes to the membership of
access control groups are not often required.

Tow types of access control groups are supported by the authentication
service: {\em primary access groups}, which can only have one member and
thus represent users or servers, and {\em secondary access groups}, which
may contain both primary and other secondary access groups as members.  Thus
the {\em protection domain\/} of a user is given by the transitive closure
of all the access groups of which he or she is an authenticated member.

\subsection{Interface}
The calls that can be made to the authentication service to manipulate
access control groups are:

\begin{quotation} \noindent {\small
\verb"create-group(private-port,group-name,group-type,owner)"\\
\hspace*{1in}$\rightarrow$ \verb"(group-id)"}

Creates a new access control group.  The private port identifies the user
making the call.  Only the system administrator is allowed to create primary
access groups.  The creator of the group is by default the {\em owner\/} of
the new group although the creator may specify that the owner is some other
group (either primary or secondary).
\end{quotation}

\begin{quotation} \noindent {\small
\verb"delete-group(private-port,group-name)"}

Deletes an access control group.  The private port identifies the user
making the call.  Only the owner or a member of the owner group is allowed
to perform the deletion.
\end{quotation}

\begin{quotation} \noindent {\small
\verb"add-to-group(private-port,group-name,add-group)"}

Adds the add-group (primary or secondary) as a member of a secondary access
group.  The private port identifies the user making the call.  Only the
owner or a member of the owner group is allowed to perform the addition.
\end{quotation}

\begin{quotation} \noindent {\small
\verb"remove-from-group(private-port,group-name,remove-group)"}

Removes the remove-group (primary or secondary) as a member of a secondary
access group.  The private port identifies the user making the call.  Only
the owner or a member of the owner group is allowed to perform the removal.
\end{quotation}

\begin{quotation} \noindent {\small
\verb"list-members(group-name)"}

Lists the members of a secondary access group.
\end{quotation}

\begin{quotation} \noindent {\small
\verb"list-memberships(group-name)"}

Lists the groups of which an access group is a member.
\end{quotation}

\end{document}
