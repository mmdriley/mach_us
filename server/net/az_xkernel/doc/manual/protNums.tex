% 
% protNums.tex
%
% $Revision: 1.4 $
% $Date: 1992/02/12 23:53:00 $
%

\section{Protocol Tables}
\label{protnum}

The \xk{} determines relative protocol numbers by reading protocol
table files at boot time.  Here is an example of a file defining a
protocol table (see section \ref{relprotnum} for an operation that
queries this table):

\begin{verbatim}
#
# prottbl
#
# This file describes absolute protocol id's 
# and gives relative protocol numbers for those
# protocols which use them
#
# x-kernel v3.2

eth         1    
{
        ip         x0800
        arp        x0806
        rarp       x8035
        #
        # ethernet types x3*** are not reserved
        #
        blast      x3001
}
ip          2
{
        icmp       1
        tcp        6
        udp        17
        #
        # IP protocol numbers n, 91 < n < 255, are unassigned
        #
        blast      101
}
arp         3
rarp        4
udp         5
tcp         6
icmp        7
blast       8
sunrpc      9

\end{verbatim}

\noindent Each protocol has an entry of the form:

\medskip

\begin{quote}
{\tt name   idNumber    [ \{  hlp1  relNum1  hlp2 relNum2  ... \} ] }
\end{quote}

\medskip

\noindent where the idNumber uniquely identifies each protocol.

There are two ways for a protocol to define its relative protocol
number space.  Using the first technique (e.g., ETH and IP), the
protocol explicitly indicates which protocols may be configured above
it and what their relative numbers are.  Use of this numbering scheme
is indicated by the presence of the optional hlp list after the
protocol's idNumber.

With the second technique, the protocol does not explicitly name its
allowed upper protocols, but will implicitly use each protocol's
unique idNumber as its relative protocol number.  BLAST, for example,
employs this technique and would use SUNRPC's idNumber 9 as its
relative protocol number above BLAST.  Protocols using this second
scheme use a 4-byte field in their headers for the relative protocol
number.

The latter technique clearly allows more flexibility in how protocols
may be composed.  As flexible composability is one of the goals of the
\xk{}, all new protocols written in the \xk{} should use this absolute
numbering scheme.

All protocols must have an entry in the protocol table.  If you are
writing a new protocol, you will probably want to define an auxiliary
protocol table which assigns a temporary idNumber to your new protocol
and then configure your kernel to read in both the system table and
your auxiliary table (see section \ref{compose} for configuration
examples.)  If you need to configure your new protocol above an
existing protocol which uses explicit numbering, you can augment the
table for the existing protocol as in this example:

\begin{verbatim}
#
# Auxiliary protocol table 
#

yap	1000

ip	2 	{
        yap	200
}
\end{verbatim}

Here we define our new protocol YAP and indicate that it has protocol
number 200 relative to IP.  Note that when augmenting the tables of
other protocols, the idNumber of the other protocol must match its
number in the system file.

The \xk{} runs consistency checks on the protocol tables and will give
error messages and abort in the presence of inconsistencies.



