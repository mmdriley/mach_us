%
% running.tex
%
% $Revision: 1.16 $
% $Date: 1993/02/05 22:49:05 $
%

\section{Running a Kernel}
\label{running}

This section describes the procedure for running a kernel
in each platform. 

\subsection{SunOS4}
\label{running_sunos}

How the \xk{} runs on top of SunOS depends on what ``device drivers''
are configured into the kernel. In the standard release, the ethernet
driver is really a simulator that uses UDP to send packets between
hosts. In this case, if you implement IP and UDP \underbar{within} the
\xk{}, then the simulated UDP packets you send are, in turn, encapsulated
in real UDP packets. This means that protocols and programs built on top
of simulated UDP can only talk to their peers in other simulators; they
cannot communicate with ``real'' versions of those protocols running 
stand-alone, for example, on a Unix machine.  

An alternative is to write an ETH protocol that sends and receives
real ethernet packets, perhaps by using the STREAM interface to the
driver available in SunOS 4.1. Similarly, one could imagine writing a
UNIXIP device driver that sends and receives real IP packets using a
raw datagram socket. The rest of this section assumes the simulator
version of ETH that runs on top of the UDP protocol provided by the
SunOS kernel.

As the result of configuring a kernel (section \ref{config}), a file named
{\sanss xkernel} should exist in your current directory ({\sanss 
/usr/xkernel/build/yap}). While in this directory, create a sub-directory
for each host you want to simulate; e.g.,

\begin{quote}
\begin{tt}
mkdir client\\
mkdir server
\end{tt}
\end{quote}

In each of these directories, create a file named {\sanss rom}.  This file
contains information that the simulated host ``reads in from ROM''. The
following are example {\sanss rom} files that might be contained in
{\sanss client} and {\sanss server}, respectively.

\begin{quote}
\begin{tt}
simeth	3050\\
arp		128.10.5.54	192.12.69.1  3050\\
arp		128.10.5.23	192.12.69.1  3051\\
\end{tt}
\end{quote}

\begin{quote}
\begin{tt}
simeth	3051\\
arp	128.10.5.54     192.12.69.1  3050\\
arp	128.10.5.23	192.12.69.1  3051\\
\end{tt}
\end{quote}

\noindent
The first field of each line indicates the protocol for which the
information is relevant.

For the simeth entry, the second field indicates the UDP port number
on which this simulated host will receive network packets. A unique
port number must be used for each simulated host running on any given
real processor.  Simulated hosts running on different processors can
use the same port number. (In this example, the two simulated hosts
run on the same real processor (192.12.69.1) and use different UDP port
numbers: {\sanss 3051} and {\sanss 3050}.)  Note that the name of the
ethernet protocol appears exactly as it does in the {\sanss graph.comp}
file.

For the arp entries, each line corresponds to a simulated IP host.
The second field is the simulated IP host number, the third field is
the actual IP host number where the simulator runs, and the third
field is the simulator's UDP port number.  
Note that the simulated IP host numbers
do not necessarily correspond to the real IP address of the machine on
which the simulated host is running.

Each simulated host runs as a separate Unix process.  If you using a
windowing user interface, then you will probably want to start each
process in a separate window.  You will also want to have each {sanss
rom} file in a subdirectory below the xkernel image that is being
tested.  For each simulated host, open a shell command window, {\sanss
cd} to the sub-directory that contains that host's {\sanss rom}
file---e.g., {\sanss cd client}---and then type {\sanss ../xkernel}.
To stop a simulated host, type {\sanss DELETE} or {\sanss CTRL-C}.

\subsection{Mach 3}

As the result of configuring a kernel (section \ref{config}), a file
named {\sanss xkernel} should exist in your current directory ({\sanss
/usr/xkernel/build/yap}).  You may optionally create a file named
{\sanss rom} in this directory.  This file contains information that
the host ``reads in from ROM''.  This is an example {\sanss rom} file:

\begin{quote}
\begin{tt}
ethdrv/SE0      priority 200\\
arp		192.12.69.88  8:0:20:1:51:E6\\
arp		192.12.69.89  8:0:20:1:50:40\\
\end{tt}
\end{quote}

The first field of each line indicates the protocol for which the
information is relevant.

Each ARP entry corresponds to an initial binding between the IP
address and the ethernet address.  These entries 
are not necessary if you have hosts on your network which will
respond to your RARP and ARP requests.

The ethdrv entry indicates the priority of the Mach packet filter used by
the \xk{}.  The default priority, 200, causes
the \xk{} to grab ethernet packets and not allow
the native Mach networking code to see them.  This allows the \xk{} to
fully communicate with other hosts, but makes it difficult to run
\xk{} tests from a remote host.  

By lowering the filter priority below 100, both Mach and the \xk{} see
all ethernet packets.  This is useful for allowing normal network
activity while running \xk{} tests, but can result in troublesome
interaction between the \xk{} protocols and the Mach protocols running
on the same host.  If you want to lower the filter priority, we
recommend either

\begin{itemize}

\item
Configuring your \xk{} without TCP, SUNRPC and any other protocols which
might cause interaction problems

or

\item
modifying your protocol table file to create an alternate protocol
number for IP relative to ETH (the default is x0800.)  In this case,
of course, the \xk{} protocols will only be able to communicate with 
other hosts running the \xk{} with the althernate protocol table.

\end{itemize}

Since the \xk{} reads from the ethernet device, it needs to run with
root uid.  You will need to either su to root before running it or run
it setuid root.  Some Unix servers will allow members of group {\em kmem}
to read the device; in this case members of the group need not su to root.

See the man page for the Mach ethernet driver {\em ethdrv} in appendix
\ref{protman} for
additional configuration requirements.

Since the \xk{} on the Mach platform does not respond well to
CTRL-C's, we recommend running it in the background so you can easily
send it a SIGKILL with the 'kill' command.


\subsection{Mach 3 \xk{} in the Mach kernel}

For this configuration, the ethernet driver name should be
``xklance/SE0.''  At this time no options are defined for the ethernet
driver.  The arp entries are the same as described above.


\subsection{Running Test Suites}
If you have configured protocols from the {\sanss test} directory into
the kernel, you can run any subset of them by putting the test names
onto the command line, e.g.:

\begin{quote}
\begin{tt}
xkernel -testip -testudp
\end{tt}
\end{quote}

With no command line test selections, all of the configured test protocols
will run.

Some of the tests run with a ``client side'' and a ``server side''.
If the macros SITE\_CLIENT* and SITE\_SERVER* in xkernel/etc/site.h
reflect the client and server you will be using, no additional
parameters are necessary.  If you are running the client and server
on different hosts, the server should be started up with a ``-s''
flag:

\begin{quote}
\begin{tt}
xkernel -testip -testudp -s
\end{tt}
\end{quote}

The client side
must also be told the IP address of the server host test peer.  This can be
done with the ``-c'' command line option, e.g.:

\begin{quote}
\begin{tt}
xkernel -testip -testudp -c192.12.69.1
\end{tt}
\end{quote}



\subsection{Troubleshooting}

Many of the problems encountered when running an \xk{} turn out to be
configuration problems.  Here are a few common symptoms and some
things you might want to check if these symptoms occur (setting the
debugging variable {\tt traceprotocol} to {\tt TR\_EVENTS} or higher
can be very helpful in identifying problems):

\begin{itemize}

\item{}
\xk{} hangs in arp\_init. 

ARP's initialization routine will not return until it has discovered
the binding for its local IP address.  If the \xk{} hangs in
arp\_init, ARP is probably sending out RARP requests which are not
being answered.  Multiple warning messages of the form:

\begin{quote}
\begin{tt}
ARP: Could not get my IP address for interface eth (still trying)
\end{tt}
\end{quote}

are an indicator of this problem.

If you are running the sunos simulator, you must have
an ARP binding in your ROM file for your local host (see section
\ref{running_sunos}.)  On other platforms, an ARP binding for the local
host is not necessary if another host on your network will respond to
RARP broadcasts.  If you do not have such a host (or if it is not
responding for whatever reason), adding a local binding to your ROM
file should fix the problem.

\item{}
Messages sent out on one host are never received on the destination host.

Check your ROM file.  If it contains an initial binding for the
destination host which is incorrect, the destination host will not see
packets from the sending host.  

\item{}
An xOpen hangs for a while and then fails.

An open may fail if ARP can not resolve the IP address of the
destination (turning ARP tracing on can help identify this problem.)
ARP requests should never be sent on the sunos simulator platform.  If
you are running on the sunos platform and you see ARP requests being
sent, check the ROM file on the sending host and make sure there is an
ARP binding for the destination host.


\item{}
The \xk{} aborts before a protocol's init routine is called.

If the \xk{} can't find a protocol number for your protocol in any of
its tables, it will abort before calling the protocol's initialization
routine.  You will need to add an entry for your protocol in one of
the tables (see section \ref{protnum}.)

\item{}
Messages get to the destination host but never make it up to the
appropriate protocol.

Make sure that the source and destination hosts are running with
identical protocol table entries for the protocol in question.  If the
numbers are different, messages won't get to the appropriate protocol
on the destination host.

\end{itemize}