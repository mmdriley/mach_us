% 
% protDesign.tex
%
% $Revision: 1.1 $
% $Date: 1992/01/03 18:19:04 $
%

\section{Protocol Design}

This section has some protocol design rules which protocol writers
should follow in order to develop ``well-behaved'' protocols that
interact properly with the other protocols in a kernel.

\subsection{Symmetry}

While designing a protocol, one needs to determine whether the
protocol will be {\sl symmetric}.  In a symmetric protocol, all
sessions process both incoming and outgoing packets for the
connection.  Asymmetric protocols, typically request-reply RPC
protocols, have different kinds of sessions for clients and servers.

If an upper protocol assumes that its lower protocol is symmetric, it
can push a message on any lower session from which it receives a
message.  Upper protocol code becomes more complicated (and less
efficient) if it can not make assumptions of symmetry.  

So in the interests of simplifying protocol code, all asynchronous
protocols should be symmetric.  Synchronous (RPC) protocols may be
asymmetric.  Boundary protocols (such as SUNRPC and CHAN) which are
asymmetric but expect a symmetric protocol below them will be so
indicated in their manual pages in appendix \ref{protman}.


\subsection{Thread Turnaround}

Protocols should refrain from taking threads which are shepherding
outgoing messages down the protocol stack and turning them around to
accompany messages traveling up the protocol stack.  Since protocols
{\em are} allowed to reverse thread direction from incoming to outgoing,
allowing turnaround from outgoing to incoming could lead to a thread
caught in a recursive loop.

If an outgoing thread needs to send a message back up, it should start
a new thread to do this.  The push routine of the ethernet protocol
({\sanss /usr/xkernel/protocols/eth}) has an example of how this is done.





