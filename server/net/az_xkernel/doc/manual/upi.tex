% 
% $RCSfile: upi.tex,v $
%
% $Revision: 1.18 $
% $Date: 1993/02/05 21:17:56 $
%

\section{Uniform Protocol Interface}

Each \xk{} protocol is encapsulated in a uniform protocol interface.
The suite of protocols configured into the system form a {\em protocol
graph} and the collection of currently opened sessions (connections)
form a {\em session graph}.

\subsection{Type Definitions}

\subsubsection{XObj}

An {\em XObj} is the fundamental identifier in the system.  It can
represent either a protocol or a session.  Fields which can be read or
written by the programmer are so indicated.

\begin{tabbing}
xxxx \= xxxx \= xxxxxxxxxxx \= xxxxxxxxxxx \= \kill
\>{\sem typedef struct} {\bold xobj} \{\\
\>\>  {\sem XObjType}  	\>type;\\
\>\>  {\sem char} 	\>*fullName;\>{\smallfont /* the name given in graph.comp, e.g., ``ethdrv/SE0''  */}\\
\>\>  {\sem char} 	\>*name;\>{\smallfont /* the protocol name, e.g., ``ethdrv'' */}\\
\>\>  {\sem char} 	\>*instName;\>{\smallfont /* the instance name, e.g., ``SE0'' */}\\
\>\>  {\sem char} 	\>*state;\>{\smallfont /* Readable/writable -- Owned by the session/protocol */}\\
\>\>  {\sem Bind} 	\>binding;\>{\smallfont /* Readable/writable */}\\
\>\>  {\sem int} 	\>rcnt;\>{\smallfont /* readable */}\\
\>\>  {\sem int} 	\>id;\>{\smallfont /* readable */}\\
%\>\>  {\sem int} 	\>instance;\>{\smallfont /* readable */}\\
\>\>  {\smallfont /* Interface functions */} \\
\>\>  {\sem Pfo} 	\>open;\>{\smallfont /* protocols only */}\\
\>\>  {\sem Pfk} 	\>closedone;\>{\smallfont /* protocols only */}\\
\>\>  {\sem Pfk} 	\>openenable;\>{\smallfont /* protocols only */}\\
\>\>  {\sem Pfk} 	\>opendisable;\>{\smallfont /* protocols only */}\\
\>\>  {\sem Pfk} 	\>opendone;\>{\smallfont /* protocols only */}\\
\>\>  {\sem Pfk}  	\>demux;\>{\smallfont /* protocols only */}\\
\>\>  {\sem Pfk}  	\>calldemux;\>{\smallfont /* protocols only */}\\
\>\>  {\sem Pfk}  	\>shutdown;\>{\smallfont /* protocols only */}\\
\>\>  {\sem Pfk}  	\>pop;\>{\smallfont /* sessions only */}\\
\>\>  {\sem Pfk}  	\>callpop;\>{\smallfont /* sessions only */}\\
\>\>  {\sem Pfh}  	\>push;\>{\smallfont /* sessions only */}\\
\>\>  {\sem Pfk}  	\>call;\>{\smallfont /* sessions only */}\\
\>\>  {\sem Pfk}  	\>close;\>{\smallfont /* sessions or protocols */}\\
\>\>  {\sem Pfi}  	\>control;\>{\smallfont /* sessions or protocols */}\\
\>\>  {\sem Pfk}  	\>duplicate;\>{\smallfont /* sessions only */}\\
\>\>  {\sem int}  	\>numdown; \>{\smallfont /* readable */}\\
\>\>\>{\smallfont /* down info is owned by the protocol for its sessions */}\\
\>\>  {\sem int} 	\>downlistsz;\\
\>\>  {\sem unsigned int} \>idle;\\
\>\>  {\sem struct xobj} \>*down[8];\\
\>\>  {\sem struct xobj} \>**downlist;\\
\>\>  {\sem struct xobj} \>*myprotl;\> {\smallfont /* readable -- a session is an instance of this protocol */}\\
\>\>  {\sem struct xobj} \>*up; \>{\smallfont /* for sessions only */}\\
\>\>  {\sem struct xobj} \>*hlpType;\\
\>\} *{\bold XObj};
\end{tabbing}

If you think of the \xk{} as implementing protocol and session graphs,
then each {\em XObj} represents a node in one of the graphs.
Furthermore, an object's {\em down} vector represents graph edges;
they contain pointers to the {\em XObjs} that are below the object in
the protocol or session graphs. The fields {\em myprotl} and {\em up}
link a session to the protocols that own and created the session,
respectively.

Note that the one aspect of Version 3.2 that differs most from
descriptions of the $x$-kernel found in the literature is that {\it
protocol} and {\it session} objects have been unified, and replaced
with a single type {\em XObj}.  As a consequence, routines that really
require a session or a protocol object check the type field of the
{\em XObj} object, where session objects have the type {\em Session}
and protocol objects have the type {\em Protocol}.

\subsubsection{Enable Objects}
\label{enable_objects}

The {\em Enable} object is provided for use by the programmer in
maintaining maps that associate identifiers and protocols or sessions.
A pointer to an {\em Enable} object is often used as the internal
identifier in {\em mapBind} calls.  An {\em Enable} object has a field
for reference counting. Protocol writers will find this useful in
handling {\em xOpenEnable} calls with identical participants---the
calls are redundant with respect to session creation---but they must
be reference counted in order to properly handle {\em xOpenDisable}
calls.

\begin{tabbing}
xxxx \= xxxx \= xxxxxxxxxxx \= xxxxxxxxxx \=\kill
\>{\sem typedef struct} {\bold xenable}  \{\\
\>\>  {\sem XObj}         \>hlpRcv;\>\smallfont{/* upper protocol */}\\
\>\>  {\sem XObj}         \>hlpType;\>\smallfont{/* upper protocol */}\\
\>\>  {\sem Bind} 	\>binding;\>\smallfont{/* from mapBind */}\\
\>\>  {\sem int} 	\>rcnt;\>\smallfont{/* A use count */}\\
\>\} {\bold Enable};
\end{tabbing}

\subsubsection{Return Values}

Most routines have a return value type of {\em xkern\_return\_t},
which is either {\em XK\_SUCCESS} or {\em XK\_FAILURE}.  Routines that
return type {\em XObj} have a failure value of {\em ERR\_XOBJ}.  Some
message handling routines use type {\em xmsg\_handle}; see section
\ref{xPush}.  Severe error conditions will result in console error
messages and the termination of the $x$-kernel.

\subsubsection{Function Types}

The following typedefs are used in the {\em XObj} structure and
throughout this document:

\begin{quote}
{\sem typedef xkern\_return\_t } ( * Pfk )();\\
{\sem typedef void } ( * Pfv )();\\
{\sem typedef struct xobj } ( * Pfo )();\\
{\sem typedef xmsg\_handle\_t } ( * Pfh )();\\ 
{\sem typedef int } ( * Pfi )();
\end{quote}

\subsection{Protocol-Protocol Operations}

This section defines the operations that protocols (and sessions)
invoke on each other. In general, each of the operations invokes a
corresponding operation in the target protocol or session.  For
example, an {\em xOpen} call will result in the invocation of a
protocol specific open routine, e.g.  {\em udp\_open}.  For each
operation, we give the interface to both the generic $x$-kernel
operation and an example protocol-specific procedure that implements
the generic operation.  Although nearly the same, the specification
for the generic operation and the specification for the
protocol-specific routine typically differ in that a {\it self}
pointer is passed to the protocol-specific routine.

\subsubsection{xOpen}

\noindent The {\em xOpen} function is used by high-level protocol {\em
hlpRcv} to actively open a session associated with low-level protocol
{\em llp} on behalf of high-level protocol {\em hlpType}. Typically,
{\em hlpRcv} and {\em hlpType} refer to the same protocol 
(see section \ref{hlpdesc}.)  
The {\em participants} 
argument is a list of addresses for each participant in
the communication.  For this, and all calls returning type {\em XObj},
a return value of {\em ERR\_XOBJ} indicates failure, and this must be
checked by all callers before using the return value.

Caveat: the high-level protocol will use its {\em self} object as the
first argument in {\em xOpen}, and the lower-level protocol object as
the third argument.  The lower-level's open routine will see its own
{\em self} object as the first argument, and the high-level protocols
as the second and third arguments.  This reversal of argument order
preserves the convention that the current protocol's self object is
the first argument in the list.

\medskip
{\sanss Generic: }
{\sem XObj} {\bold xOpen}({\sem XObj} {\caps  hlpRcv}, {\sem XObj} {\caps  hlpType}, {\sem XObj} {\caps  llp}, {\sem Part} *{\caps participants})
\medskip

{\sanss Specific: }
{\sem XObj} {\bold udp\_open}({\sem XObj} {\caps self}, {\sem XObj} {\caps  hlpRcv}, {\sem XObj} {\caps  hlpType}, {\sem Part} *{\caps participants})

\subsubsection{xOpenEnable}

\noindent Used by high-level protocol {\em hlpRcv} to passively open a 
session associated with low-level protocol {\em llp} on behalf of
high-level protocol {\em hlpType}. As with {\em xOpen}, {\em hlpRcv}
and {\em hlpType} usually refer to the same protocol. A passive open
indicates a willingness to accept connections initiated by remote
participants.  A session is not actually returned, but the low-level
protocol, by convention, ``remembers'' this enabling, and later calls
the high-level protocol's {\em xOpenDone} operation to complete the
passive open.  {\em participants} is an ordered list of addresses for
each participant for which the communication has been enabled. In most
cases, it contains only a single element: the address of the local
participant.  A return value of {\em XK\_FAILURE} indicates failure.

The lower-level protocol generally ``remembers'' an invocation of its
{\em xOpenEnable} operation by binding an {\em Enable} object to the
participant information using {\em mapBind}.
\medskip

{\sanss Generic:} {\sem xkern\_return\_t} {\bold xOpenEnable} (
{\sem XObj} {\caps hlpRcv}, {\sem XObj} {\caps hlpType}, 
{\sem XObj} {\caps llp}, {\sem Part} *{\caps p}
)
\medskip

{\sanss Specific:} {\sem xkern\_return\_t} {\bold udp\_openenable} (
{\sem XObj} {\caps self}, 
{\sem XObj} {\caps hlpRcv}, {\sem XObj} {\caps hlpType}, 
{\sem Part} *{\caps p})

\subsubsection{xOpenDisable}

\noindent Used by high-level protocol {\em hlpRcv} to undo the effects
of an earlier invocation of {\em xOpenEnable}. The contents of the
{\em participant} argument, { \em hlpRcv } and { \em hlpType } must be
the same as the ones given to {\em xOpenEnable}.\medskip

{\sanss Generic:} {\sem xkern\_return\_t} {\bold xOpenDisable}(
{\sem XObj} {\caps hlpRcv}, {\sem XObj} {\caps hlpType}, 
{\sem XObj} {\caps  llp}, {\sem Part}*{\caps p})
\medskip

{\sanss Specific:} {\sem xkern\_return\_t} {\bold udp\_opendisable}(
{\sem XObj} {\caps self}, 
{\sem XObj} {\caps hlpRcv}, {\sem XObj} {\caps hlpType}, 
{\sem Part}*{\caps p})


\subsubsection{xOpenDisableAll}

\noindent 
Used by high-level protocol {\em hlpRcv} to 
inform low-level protocol {\em llp} that all previous
openEnables made by {\em hlpRcv} should be removed.  

\medskip

{\sanss Generic:} {\sem xkern\_return\_t} {\bold xOpenDisableAll}
( {\sem XObj} {\caps hlpRcv}, {\sem XObj} {\caps  llp} )
\medskip

{\sanss Specific:} {\sem xkern\_return\_t} {\bold udp\_opendisableall}
( {\sem XObj} {\caps self}, {\sem XObj} {\caps hlpRcv} )


\subsubsection{xOpenDone}

\noindent Used by low-level protocol to inform high-level
protocol that session ({\em s}) has now been created corresponding to
an earlier {\em xOpenEnable} on behalf of {\em hlpType}.  

\begin{quote}
{\em 3.1 compatibility note:} This is like 3.1's x\_callopendone
without participants or hlp; the original x\_opendone no longer exists.
\end{quote}
\medskip


{\sanss Generic:} {\sem xkern\_return\_t} {\bold xOpenDone} (
{\sem XObj} {\caps hlpRcv}, {\sem XObj} {\caps s}, {\sem XObj} {\caps llp} ) 
\medskip

{\sanss Specific:} {\sem xkern\_return\_t} {\bold udp\_opendone}
({\sem XObj} {\caps self}, {\sem XObj} {\caps lls}, {\sem XObj} {\caps llp}, 
{\sem XObj} {\caps hlpType}
)

\subsubsection{xPush}\label{xPush}

\noindent Used by a high-level protocol that opened session {\em s}
to pass the message {\em msg} down through that session.  The return
value is an opaque handle on the message that was sent.  This handle
may be used to identify this message in subsequent {\em xControl} operations.
The message handle may also take one of two special values:
a return value of {\em XMSG\_ERR\_HANDLE} indicates general failure, 
while {\em XMSG\_ERR\_WOULDBLOCK} indicates that a session in
non-blocking mode would normally have blocked the push.

\begin{quote}
{\em 3.1 compatibility note:} 
The reply message arg from 3.1 has been eliminated.  Use xCall if a
reply message is expected.
\end{quote}
\medskip

{\sanss Generic:} {\sem xmsg\_handle\_t} {\bold xPush} ({\sem XObj} {\caps s}, {\sem Msg} *{\caps msg})
\medskip

{\sanss Specific:} {\sem xmsg\_handle\_t} {\bold udp\_push} ({\sem XObj} {\caps self}, {\sem Msg} *{\caps msg})

\subsubsection{xCall}

\noindent Similar to {\em xPush} except that a ``reply'' message
may be returned through the argument {\em reply\_msg}.  Used with
synchronous (RPC-like) protocols.  Because the lower protocol
typically retains no state for the request message after {\em xCall}
returns, a message handle is not returned.  The message structure for
the reply must be initialized; see {\em msgConstructEmpty}.
\medskip

{\sanss Generic:} {\sem xkern\_return\_t} {\bold xCall} ({\sem XObj} {\caps s}, {\sem Msg} *{\caps msg}, {\sem Msg} *{\caps reply\_msg})
\medskip

{\sanss Specific:} {\sem xkern\_return\_t} {\bold udp\_call} ({\sem XObj} {\caps s}, {\sem Msg} *{\caps msg}, {\sem Msg} *{\caps reply\_msg})

\subsubsection{xPop}\label{xPop}

\noindent Used by a protocol to pass an incoming message {\em msg} up to 
session {\em s} for processing, and to indicate the lower-level
session from which the message was received ({\em lls}).  This calls
the pop routine of the session {\em s} and increments the session
reference count (see section \ref{refcnt}).  This call is invoked by a
protocol on one of its own sessions.

The {\em hdr} argument is passed through the infrastructure directly
to the protocol-specific routine.  It is typically used to pass the
header (which the {\em demux} routine used to find the session)
to the session's {\em pop} routine.

\medskip

{\sanss Generic:} {\sem xkern\_return\_t} {\bold xPop} 
({\sem XObj} {\caps s}, {\sem XObj} {\caps lls}, {\sem Msg} *{\caps msg},
 {\sem void *} {\caps hdr})
\medskip

{\sanss Specific: }{\sem xkern\_return\_t} {\bold udp\_pop} 
({\sem XObj} {\caps self}, {\sem XObj} {\caps lls}, {\sem Msg} *{\caps msg},
 {\sem void *} {\caps hdr})


\subsubsection{xCallPop}

When a  synchronous (RPC-like) protocol will  be demuxing a message to
an asynchronous  protocol, {\em xCallPop}  can  be  used  to allow the
upper protocol to return a message. This reply message may be the same
one passed to the synchronous protocol via {\em xCallDemux}.
\medskip

{\sanss Generic:} {\sem xkern\_return\_t} {\bold xCallPop} 
({\sem XObj} {\caps s}, {\sem XObj} {\caps lls}, 
{\sem Msg} *{\caps msg},  {\sem void *} {\caps hdr},
{\sem Msg} *{\caps rmsg})
\medskip

{\sanss Specific:} {\sem xkern\_return\_t} {\bold udp\_callpop} 
({\sem XObj} {\caps self}, {\sem XObj} {\caps lls}, 
{\sem Msg} *{\caps msg},  {\sem void *} {\caps hdr},
{\sem Msg} *{\caps rmsg})

\subsubsection{xCloseDone}

\noindent Used by a low-level protocol to inform the high-level protocol 
that originally opened session {\em s} that the session has been closed by
a peer participant.
\medskip

{\sanss Generic:}{\sem xkern\_return\_t} {\bold xCloseDone} ({\sem XObj} {\caps s})
\medskip				

{\sanss Specific:} {\sem xkern\_return\_t} {\bold udp\_closedone} ({\sem XObj} {\caps s})

\subsubsection{xClose}

\noindent Decrements the reference count of an {\em XObj} (see section
\ref{refcnt}), calling the object's close function {\em only} if the
reference count is zero. This call applies to {\em XObj}s that represent
both protocols and sessions, but the implications of invoking {\em xClose}
on a protocol is not fully determined at this time.

\begin{quote}
{\em 3.1 compatibility note:} Replaces xCloseProtl and xCloseSession.
Note that in 3.1, the {\em XObj}'s close function was called every
time xCloseProtl or xCloseSession was called.
\end{quote}
\medskip

{\sanss Generic:} {\sem xkern\_return\_t} {\bold xClose} ({\sem XObj} {\caps object})
\medskip

{\sanss Specific:} {\sem xkern\_return\_t} {\bold udp\_close} ({\sem XObj} {\caps object})

\subsubsection{xControl}

Used by one {\em XObj} to act upon another {\em XObj} for retrieving
information or for setting processing parameters.  The operation code
{\em opcode} identifies the action; {\em buf} is a character buffer
from which an argument is retrieved and/or into which a result is
placed; and {\em len} is the length of the buffer.  There are two
``classes'' of operations: standard ones that may be implemented by
more than one protocol, and protocol-specific ones.  A full discussion
of control operation codes is in section \ref{control}.

\begin{quote}
{\em 3.1 compatibility note:} The routines controlprotl and controlsessn
no longer exist.
\end{quote}
\medskip

{\sanss Generic:} {\sem int} {\bold xControl} ({\sem XObj} {\caps object}, 
{\sem int} {\caps opcode}, {\sem char} *{\caps buf}, {\sem int} {\caps len})
\medskip

{\sanss Specific:} {\sem int} {\bold udp\_control} ({\sem XObj} {\caps self}, {\sem int} {\caps opcode}, {\sem char} *{\caps buf}, {\sem int} {\caps len})

\subsubsection{xDemux}

\noindent Used by low-level session {\em s} to pass message {\em msg}
to the high-level protocol that created it.  The higher-level protocol
is determined from the session object.  The higher-level protocol
demux routine should find the appropriate session, creating it if
necessary, and {\em xPop} the message to the session. See
\ref{xCreateSessn} for guidelines on when session creation is
appropriate.

The called routine is presented with its protocol object, the session
object from the lower-level protocol, and the message.
\medskip

{\sanss Generic:} {\sem xkern\_return\_t} {\bold xDemux} ({\sem XObj} {\caps s}, {\sem Msg} *{\caps msg})
\medskip

{\sanss Specific:} {\sem xkern\_return\_t} {\bold udp\_demux} ({\sem
XObj} {\caps self}, {\sem XObj} {\caps lls}, {\sem Msg} *{\caps msg})

\subsubsection{xCallDemux}

This call is like {\em xDemux} but allows an argument to contain a
return message. Used with synchronous (RPC-like) protocols.
\medskip

{\sanss Generic:} {\sem xkern\_return\_t} {\bold xCallDemux} ({\sem XObj} {\caps s}, {\sem Msg} *{\caps msg}, {\sem Msg} *{\caps rmsg})
\medskip

{\sanss Specific:} {\sem xkern\_return\_t} {\bold upd\_calldemux}
({\sem XObj} {\caps self}, {\sem XObj} {\caps lls}, {\sem Msg} *{\caps msg}, {\sem Msg} *{\caps rmsg})

\subsubsection{xDuplicate}
\label{duplicate}

Increments the reference count of session {\em s}.  This can be used
to create a permanent handle on the {\em XObj} from a temporary handle,
or to create a new equivalent handle from an existing handle. For a
full discussion of session reference counts, see section \ref{refcnt}.
\medskip

{\sanss Generic:} {\sem xkern\_return\_t} {\bold xDuplicate} ({\sem XObj}  {\caps s})
\medskip

{\sanss Specific:} {\sem xkern\_return\_t} {\bold udp\_duplicate} ({\sem XObj}  {\caps self})



\subsubsection{xShutDown}

Called once for each protocol object to indicate imminent death of the
\xk{}.  A protocol may choose to take actions to shut down in a
graceful manner.  In particular, boundary protocols may choose to
sever their connections to the outside world in an orderly manner on
receipt of this call.

Protocols should generally not invoke this operation on other
protocols or on themselves.  It is typically invoked by the \xk{}
infrastructure itself.

\medskip

{\sanss Generic:} {\sem xkern\_return\_t} {\bold xShutDown} 
({\sem XObj} {\caps protl} )
\medskip

{\sanss Specific:} {\sem xkern\_return\_t} {\bold udp\_shutdown} 
({\sem XObj}  {\caps self})



\subsection{Graph Manipulation Operations}
				
Unlike the previous set of operations---which protocols invoke on each
other to open/close connections and to send/receive messages---the
operations defined in this section actually manipulate the protocol
and session graphs; i.e, create nodes and edges. These operations
are either called by the \xk{} at start-up time to create and link
together protocol objects, or by protocols at runtime to create and
link together session objects.

\subsubsection{xCreateSessn}\label{xCreateSessn}

Called by protocol {\em llp} to create a session that will handle data
associated with a common source/destination pair. Usually called in
response to an {\em xOpen} call, or because data has arrived with
participants that match a previous {\em xOpenEnable} call.  By
convention, a protocol will only create one session at a time for a
source/destination pair, even if there have been multiple {\em
xOpenEnable}'s that would match incoming data.

The session is initialized using information found in protocols {\em hlpRcv},
{\em hlpType} and {\em llp}.  The new session's up pointer is set to
{\em hlpRcv} (this is where upward-bound messages through this session
will be delivered.)
The count {\em downc}
indicates how many lower level sessions this session requires.  
It need be non-zero only for sessions that will store
sessions in the ``down'' vector {\em downv} (it will be zero
for device drivers and one for normal sessions.)
The initialization
function pointer {\em f} may be null; otherwise this function should
fill in the interface function pointers in the {\em XObj} structure.
These pointers are initialized to default (usually null) functions by
the system initialization code.  
\medskip

{\sem XObj} {\bold xCreateSessn}({\sem Pfv} {\caps  f}, 
{\sem XObj} {\caps  hlpRcv}, {\sem XObj} {\caps  hlpType}, 
{\sem XObj} {\caps llp}, {\sem int} {\caps downc}, 
{\sem XObj} *{\caps downv})
\medskip

{\sanss Example Init Function:} {\sem void} {\bold udp\_sessn\_init} ({\sem XObj} {\caps self})

\medskip

\subsubsection{xCreateProtl}

Called during system start-up for each protocol in the graph.  The
function {\em f} is called to initialize a protocol object. It must
have a well-known name derived from the concatenation of the protocol
name and the string ``{\em \_init}'' (e.g., {\em tcp\_init}).  The
initialization generally allocates and initializes protocol state and
fills in the interface function pointers.  Because function pointers
are initialized to null functions before {\em f} is called, only those
functions actually used by the protocol need be defined.

Use of {\em xCreateProtl} outside of initialization, e.g., to
dynamically load new protocols, is not supported at this time.

\begin{quote}
{\em 3.1 compatibility note:} Replaces x\_instantiateprotl and
x\_createprotl (the user routine) and eliminates the need for x\_getproc.
\end{quote}
\medskip

{\sem XObj} {\bold xCreateProtl} ({\sem Pfv} {\caps f}, {\sem char} *{\caps name}, {\sem char} *{\caps instName}, {\sem int} {\caps downc}, {\sem XObj} *{\caps {\caps downv}} )
\medskip

{\sanss Example Init Function:} {\sem void} {\bold udp\_init} ({\sem XObj} {\caps self})

\subsubsection{xDestroy}

Destroys protocol and session objects. It is the inverse of {\em
xCreateSessn} and {\em xCreateProtl}.  Storage for the {\em XObj} is
freed, and if the state pointer of the {\em XObj} is non-null, it is
also freed.
\medskip

{\sem xkern\_return\_t} {\bold xDestroy}({\sem XObj} {\caps object})
\medskip

\subsubsection{xGetProtlByName}

Returns a capability for a protocol object given its mnemonic name.
(See the discussion of graph.comp in section \ref{config}.)
\medskip

{\sem XObj} {\bold xGetProtlByName} ({\sem char} *{\caps name} )

\subsubsection{xSetDown}

Sets the {\em index}th member of {\em self}'s down vector to be {\em
object}.  It sets or increments the {\em XObj} field {\em numdown} as
a side effect.
\medskip

{\sem xkern\_return\_t} {\bold xSetDown} ({\sem XObj} {\caps self}, {\sem int} {\caps index}, {\sem XObj} {\caps object})

\subsubsection{xGetDown}

Returns the {\em index}th member of {\em self}'s down vector.  Returns
{\em ERR\_XOBJ} if the index is larger than the down vector.
\medskip

{\sem XObj} {\bold xGetDown} ({\sem XObj} {\caps self}, {\sem int} {\caps index})


\subsubsection{xSetUp}

Resets the up pointer of session {\em s} to {\em hlp}.  The up pointer
of a session is initialized in {\em xCreateSessn}, so {\em xSetUp}
is only used for extraordinary manipulation of the session graph.
\medskip

{\sem void} {\bold xSetUp} ({\sem XObj} {\caps s}, {\sem XObj} {\caps hlp})


\subsubsection{xGetUp}

Returns the up pointer of session {\em s}.
\medskip

{\sem XObj} {\bold xGetUp} ({\sem XObj} {\caps s})


\subsubsection{xHlpType}

Returns the {\em hlpType} argument that was used to create session
{\em s.}
\medskip

{\sem XObj} {\bold xHlpType} ({\sem XObj} {\caps s})



\subsection{ Utility Operations }

\subsubsection{xIsProtocol}

Returns true if the {\em XObj} is a protocol.
\medskip

{\sem bool} {\bold xIsProtocol} ( {\sem XObj} )



\subsubsection{xIsSession}

Returns true if the {\em XObj} is a session.
\medskip

{\sem bool} {\bold xIsSession} ( {\sem XObj} )



\subsubsection{xIsXObj}

Returns true if the {\em XObj} is either a session or a protocol
(i.e., if this returns false the {\em XObj} was either never
initialized or has been badly clobbered.)
\medskip

{\sem bool} {\bold xIsXObj} ( {\sem XObj} )



\subsubsection{xIsValidXObj}

XObjects created with xCreateSessn or xCreateProtl are kept in a
system map and removed when the object is destroyed.  
{\em xIsValidXObj} can be used to determine whether a random 
XObject handle is in this map and can thus be used safely.
\medskip

{\sem bool} {\bold xIsValidXObj} ( {\sem XObj} )


\subsubsection{xPrintXObj}

Display some information about the state of the {\sem XObj}.
\medskip

{\sem bool} {\bold xPrintXObj} ( {\sem XObj} )



\subsection{Usage Rules}

This section has some protocol design rules that protocol writers
should follow in order to develop ``well-behaved'' protocols that
interact properly with other protocols with which they might be
composed.


\subsubsection{Initializing a Protocol}

At system boot time, the \xk{} calls xCreateProtocol for each protocol 
configured into the kernel (see section \ref{config}). 
xCreateProtocol, in turn, 
calls the protocol's init routine,
where for a protocol named `yap', this initialization routine 
must be named `yap\_init'. 
Generally, this routine does the following work.

\begin{verbatim}
void
yap_init(self)
    XObj self;
{
    /* fill in protocol-specific operations */
    self->open = yap_open;
    self->demux = yap_demux;
    . . .

    /* create and initialize protocol-specific state, including maps */
    self->state = xMalloc( sizeof(struct yap_state) );
    . . .

    /* get handle for, and register with, lower-level protocol */
    llp = xGetDown(self, 0));
    partInit(&part, 1);
    partPush(part, ANY_HOST, 0);
    xOpenEnable(self, self, llp, &part);
}
\end{verbatim}



\subsubsection{hlpRcv and hlpType}
\label{hlpdesc}

The operations {\em xOpen}, {\em xOpenEnable}, and {\em xOpenDisable}
take two high-level protocols, {\em hlpRcv} and {\em hlpType}.  {\em
hlpRcv} is the protocol to which the new lower session should route
incoming messages.  The lower protocol uses {\em hlpRcv} as the hlp
argument to {\em xCreateSessn}. The lower protocol uses {\em hlpType}
to determine which messages the new session should handle.  For
example, when {\em ethOpen} is called with IP as the {\em hlpType},
ETH knows that the new session will deal with packets that have the IP
ethernet type.  The lower protocol typically determines the number
that corresponds to {\em hlpType} by using it in a call to {\em
relProtNum} (see section \ref{protnum}.)

Most protocols use their {\em self} pointer as both {\em hlpRcv} and
{\em hlpType} when making these calls, though virtual protocols (see
below) are the exception.

\subsubsection{Protocol Realms}

Although the \xk{} defines a single interface for all protocols, not
all protocols are created equal. Our experience is that protocols can
be classified into different categories, which we call realms. Chances
are, any protocol you write falls into one of the following realms. In
some cases, the realm a protocol falls into both defines a restricted
subset of the interface that the protocol implements, and the set of
protocols with which it may be composed.

\paragraph{Asynchronous Protocols}{\ }

\smallskip

\noindent Most protocols---e.g., protocols like IP, TCP, and UDP---fall 
in this category. The \xk{} supports asynchronous protocols through
the use of {\em xPush}, {\em xPop} and {\em xDemux} operations.
Asynchronous protocols are typically symmetric in the sense that the
protocol's sessions process both incoming and outgoing messages.
While it seems possible for asynchronous protocols to have asymmetric
sessions (a given session can handle only incoming or outgoing
messages, but not both), we have thus far been able to make all our
asynchronous protocols be symmetric, and we strongly encourage such
designs.  Knowing that any low-level protocols you may use are
symmetric enhances our ability to compose protocols and makes
implementing a given protocol much easier.

\paragraph{Synchronous Protocols}{\ }

\smallskip

\noindent These are RPC protocols. They are typically asymmetric in the 
sense that client-side sessions and the server-side sessions are quite
different.  The \xk{} explicitly supports synchronous/asymmetric
sessions through the use of {\em xCall}, {\em xCallPop} and {\em
xCallDemux}. Since synchronous protocols are asymmetric, {\em xCall}
is used on the client side and {\em xCallPop} and {\em xCallDeumx} are
used on the server side.

Note that some protocols lie on the boundary between the synchronous
and asynchronous realms. For example, a protocol that implements RPC
(as opposed to one that uses it) probably looks asynchronous from the
bottom (i.e., lower level protocols call its {\em xPop} routine),
but synchronous from above (i.e., higher level protocols call its
{\em xCall} routine).

\paragraph{Control Protocols}{\ }

\smallskip

\noindent These protocols support neither a {\em xPush/xPop} nor a 
{\em xCall/xCallPop} interface.  Typically, only control operations
may be performed on these protocols.  ARP and ICMP fall into this
category.


\subsubsection{Anchor Protocols}

Anchor protocols sit either at the top or the bottom of a protocol
stack and provide an interface between the \xk{} and the system in
which the \xk{} is embedded.  Top-level anchor protocols look like an
\xk{} protocol from the bottom, but provide an Application Programmer
Interface to the \xk{} (see appendix \ref{api}.)  Bottom-level anchor
protocols (e.g., device drivers) look like a protocol from the top,
but typically interface with the lower levels of the surrounding
system or with network hardware.

Writing anchor protocols involves careful synchronization of
external threads with \xk{} threads and objects.  See section
\ref{ext-threads}. 

\subsubsection{Virtual Protocols}

Virtual protocols occupy places in the protocol (and sometimes the
session) graphs, but they neither produce nor interpret network headers.
They typically make decisions about how messages should be routed
through the session graph based on participants in {\em xOpen} or on
properties of messages, such as size.

The {\em xOpen}, {\em xOpenEnable}, and {\em xOpenDisable} routines of
virtual protocols differ from those of conventional protocols.  A
virtual protocol's implementation of {\em xOpen}, for example, will
usually make an {\em xOpen} call to its lower protocols using the {\em
hlpType} that was passed into the virtual protocol, but using its {\em
self} pointer as {\em hlpRcv}.  This allows arbitrary chains of
virtual protocols to insert their sessions between the upper and lower
conventional sessions while still passing ``type information'' from
the upper protocol to the lower protocol.

Note that virtual protocols can be either synchronous (support the
{\em xCall}/{\em xCallPop}/{\em xCallDemux} interface) or asynchronous
(support the {\em xPush}/{\em xPop}/{\em xDemux} interface).


\subsection{Default Operations}

Since many protocols' UPI operations look very similar, the \xk{}
provides some library operations which do much of the standard work of
some of the operations.  Many protocols can call these default
operations and greatly simplify their own implementations of these
routines.  


\subsubsection{defaultOpenEnable}

Binds {\em key} to an enable object with {\em hlpRcv} and {\em
hlpType}.  If a previous binding exists for the given key and {\em hlp}s,
the reference count of that enable object will be increased.  
{\em defaultOpenEnable} will fail if a previous binding exists for
this key that does not match the {\em hlp}s.

\medskip

{\sem xkern\_return\_t} {\bold defaultOpenEnable} 
( 
{\sem Map} {\caps m}, 
{\sem XObj} {\caps hlpRcv}, 
{\sem XObj} {\caps hlpType}, 
{\sem void} *{\caps key}
)


\subsubsection{defaultOpenDisable}

Undoes the effect of a previous {\em defaultOpenEnable.}
Returns failure if no appropriate enable object exists (e.g., if
nothing exists for the given key or if the {\em hlp}s don't match the
saved values in the enable object.)

\medskip

{\sem xkern\_return\_t} {\bold defaultOpenDisable} 
( 
{\sem Map} {\caps m}, 
{\sem XObj} {\caps hlpRcv}, 
{\sem XObj} {\caps hlpType}, 
{\sem void} *{\caps key}
)


\subsubsection{defaultOpenDisableAll}

Removes all enable objects bound in {\em map} with {\em hlpRcv}.
If {\em f} is non-zero, it is called with the ({\em key}, {\em
Enable} *) pair for each {\em Enable} object in the map before it is removed.


\medskip

{\sem xkern\_return\_t} {\bold defaultOpenDisableAll} 
( 
{\sem Map} {\caps m}, 
{\sem XObj} {\caps hlpRcv}, 
{\sem DisableAllFunc} {\caps f}
)

\medskip

{\sem typedef void} ({\bold *DisableAllFunc} )
(
{\sem void} *{\caps key}, 
{\sem Enable} *{\caps e}
)


\subsubsection{defaultVirtualOpenEnable}

Designed to be used by virtual protocols.  In addition to the binding
performed by {\em defaultOpenEnable}, an {\em xOpenEnable} is
performed on each lower protocol in the null-terminated array 
{\em llp} (using participants {\em p}, 
causing the lower protocols to deliver packets for 
{\em hlpType} to the virtual protocol {\em self}.  If any of these
{\em xOpenEnable}s fail, {\em defaultVirtualOpenEnable} backs out of
the entire operation.

Assumes that the passive map is keyed on {\em hlpType}.

\medskip

\begin{tabbing}

\indent
{\sem xkern\_return\_t} {\bold defaultVirtualOpenEnable} 
( \= 
{\sem XObj} {\caps self}, 
{\sem Map} {\caps m}, 
{\sem XObj} {\caps hlpRcv}, 
{\sem XObj} {\caps hlpType}, \\
\> {\sem XObj} *{\caps llp}, 
{\sem Part} *{\caps p}
)

\end{tabbing}

\subsubsection{defaultVirtualOpenDisable}

Undoes the effect of a previous {\em defaultVirtualOpenEnable}.

\medskip

\begin{tabbing}

\indent
{\sem xkern\_return\_t} {\bold defaultVirtualOpenDisable} 
( \=
{\sem XObj} {\caps self}, 
{\sem Map} {\caps m}, 
{\sem XObj} {\caps hlpRcv}, \\
\> {\sem XObj} {\caps hlpType}, 
{\sem XObj} *{\caps llp}, 
{\sem Part} *{\caps p}
)

\end{tabbing}



\subsubsection {Usage}

Here is an example of how a protocol might simplify its 
{\em enable}/{\em disable} routines using these operations:

\begin{verbatim}

static xkern_return_t
yapOpenEnable( XObj self, XObj hlpRcv, XObj hlpType, Part *p )
{
    long    key;
    PState  *ps = self->state;
    
    key = getRelProtNum(hlpType, self));
    return defaultOpenEnable(ps->passive_map, hlpRcv, hlpType, &key);
}

static xkern_return_t
yapOpenDisable( XObj self, XObj hlpRcv, XObj hlpType, Part *p )
{
    long    key;
    PState  *ps = self->state;
    
    key = getRelProtNum(hlpType, self));
    return defaultOpenDisable(ps->passive_map, hlpRcv, hlpType, &key);
}

static xkern_return_t
yapOpenDisableAll( XObj self, XObj hlpRcv )
{
    PState  *ps = self->state;

    return defaultOpenDisableAll(ps->passive_map, hlpRcv, 0);
}
\end{verbatim}
