
\section{System Installation}

This section describes how to install Version 3.2 of the $x$-kernel.
By installing the $x$-kernel we mean setting up the $x$-kernel
environment so that programmers can begin writing protocols and
configuring new kernels.

The $x$-kernel is commonly used for both research and instruction.  A
typical scenario is for a researcher (instructor) to install the
$x$-kernel for use by various network projects (students). Individual
projects (students) then build and test kernels in private working
directories. This section
describes the work the researcher (instructor) needs to do to set up
the $x$-kernel directory.  Most of the installation can be handled by
the researcher (instructor).  Only a few things need to be done by a
system administrator that has root permission, and these are clearly
documented below.

The $x$-kernel runs in several environments: as a task under
Mach3 or within the Mach kernel (on both DecStations and HP 700
workstations),  or as a user process on top of SunOS, release
4.1 or latter.

\subsection{Unpacking the {\it tar} file}

If you haven't done so already, create a new directory for installing
the $x$-kernel.  For the purpose of this report, we refer to this
directory as {\tt /usr/xkernel}.  You may chose another name, however.
{\tt cd} to this directory and type

\begin{quote}
{\tt tar xf} {\it file}
\end{quote}

\noindent where {\it file} is the disk or tape file of the $x$-kernel
distribution. This will create several subdirectories in {\tt
/usr/xkernel}. 


\subsection{Directory Structure}

The distribution contains the following directories:

\begin{description}

\item[{\tt bin}]~\\ Binaries for utility programs used to develop new
kernels.

\item[{\tt doc}]~\\ Documentation.

\item[{\tt etc}]~\\ Configuration-related data files and protocol
tables. 

\item[{\tt gen}]~\\ Source code for the in-kernel Mach platforms.
This code is automatically generated from various configuration files. 

\item[{\tt mach3}]~\\ Sources and binaries for the Mach3-based $x$-kernel.

\item[{\tt pi}]~\\ Sources and binaries for platform independent code:
the \xk{}'s object-oriented infrastructure, protocol table code, and
the message, participant, and map libraries.

\item[{\tt protocols}]~\\ Sources and binaries for all protocols supported 
in this release.

\item[{\tt sunos}]~\\ Sources and binaries for Unix-based $x$-kernel.

\item[{\tt util}]~\\ Utility programs for the \xk{}.


\end{description}

\noindent Note that those directories that contain $x$-kernel source
and binaries actually place the binaries in several sub-directories,
one for each ``kind'' of binary: {\tt DEBUGsunos} (Sun binaries,
compiled with debugging turned on), {\tt OPTIMIZEsunos} (Sun binaries,
compiled with optimization turned on), {\tt DEBUGmach3} (Mach3 MIPS
binaries, compiled with debugging turned on), {\tt OPTIMIZEmach3}
(Mach3 MIPS binaries, compiled with optimization turned on), etc.

\subsection{About Compilers}

Version 3.2 of the $x$-kernel now uses either the Gnu C compiler ({\tt
gcc}) or the standard Unix compiler ({\tt cc}).  If you do not
currently have Gnu C at your site, you can get a copy by doing
anonymous FTP to {\tt prep.ai.mit.edu}, and retrieving {\tt
pub/gnu/gcc.xtar}.


\subsection{ Configuration }

Many of the site-specific configuration options are listed in the file
{\tt xkernel/etc/site.h}.  These options should be edited to reflect
reasonable defaults for your site.  Many protocols have additional
options that can be set on a per-execution basis with entries in a
``rom file.''  These options are described in the manual page for each
protocol in appendix \ref{protman}.


\subsection{Building and Testing the System}

First edit {\tt xkernel/etc/site.h} to reflect reasonable default values for
your site.  The comments in site.h should be self-explanatory.

Installing the \xk{} consists of building a few utility programs used
to develop new kernels.  These are the procedures for building these
programs, though binary versions are distributed for most platforms
and rebuilding these programs is probably unnecessary.  For both of
these procedures, you need to define the environment variable {\tt
ARCH} to correspond to the type of binaries you want created:  
{\tt sparc}, {\tt mips}, or {\tt hp700}.

The \xk{} depends on gnumake, so you first have to make gnumake:
\medskip

{\tt cd} to {\tt /usr/xkernel/util/make} and type {\tt make}.  
\medskip

Next, put {\tt /usr/xkernel/bin/ARCH} and {\tt /usr/xkernel/bin}
in your search path, where ARCH is as described above.
They should appear before {\tt /bin} and {\tt /usr/bin} in
order to pick up gnumake before the standard Unix {\tt make}. Finally,
{\tt cd~/usr/xkernel/util} and type {\tt make~setup} followed by 
{\tt make~install}.  This will install a handful of programs used to
develop new kernels.  ({\tt make~install} may produce a warning of the
form:  {\tt make[1]: fopen: Makedep.ARCH: No such file or directory}.
The missing file will be created during this step and the warning can
be ignored.)


To test a system, build a kernel as described in section
\ref{config} which includes one or more of the test protocols 
(e.g., {\tt udptest}).  You should then be able to run your test
kernel as described in section \ref{running}.

