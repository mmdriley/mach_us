%
% config.tex
%
% $Revision: 1.16 $
% $Date: 1993/01/19 17:27:37 $
%

\section{Configuring a Kernel}
\label{config}

This section describes how to configure and build an \xk{}.

\subsection{Specifying a Protocol Graph}
\label{compose}

A new instance of the $x$-kernel is configured by specifying the
collection of protocols that are to be included in the kernel in a
file called {\sanss graph.comp}. Example {\sanss graph.comp} files can
be found in {\sanss xkernel/build/Template}.

The {\sanss graph.comp} file is divided into three sections: device
drivers, protocols, and miscellaneous configuration parameters.  The
sections are separated by lines beginning with {\sanss @;} and each
section may be empty.

The first two sections, device drivers and protocols, describe the
protocol graph to be configured into your kernel.  Device drivers and
protocols are described by the same types of entries, as in the
example:

\let\tt=\COURIERtt
\begin{verbatim}
name=yap files=yap,yap_clnt,yap_srvr dir=yap, protocols=ip,eth trace=TR_MAJOR_EVENTS;
\end{verbatim}
\let\tt=\CMRtt

The first field gives the protocol's name.  The rest of the fields are
optional and may occur in any order.  The {\sanss dir} and {\sanss files}
fields describe the names and locations of the source files used to
build the protocol.  Files are specified without extensions.  If no
dir entry is specified, {\em compose} will search for the source files in the
current build directory, a subdirectory in the current build directory
with the protocol's name, and finally in the system protocol library.
An empty {\sanss files} entry defaults to a single {\sanss .c} file with the
protocols name.  An entry for a protocol from the library should
contain neither a {\sanss dir} nor a {\sanss files} entry.

The {\sanss protocols} field indicates the protocols below the current
protocol in the graph.  When this field contains multiple protocols, order is
significant (the lower protocols will be loaded into the upper
protocol's down vector in the order in which they are listed.)
A protocol which expects multiple protocols below it will describe the
expected semantics of the lower protocols in its manual page in
appendix \ref{protman}.

The {\sanss trace} field defines the debugging level used in trace
statements depending on the protocol variable {\sanss traceyapp}.  See
section \ref{kdebug} for the symbolic names associated with trace levels.

Multiple instantiations of protocols are supported by using a ``/''
character after the protocol name to add an extension.  In the following
example, two instantiations of ``yap'' are indicated, one over ``ip'' and
one over ``simeth,'' and both are used by the ``prt'' protocol.  In
addition, the simeth protocol is instance ``0'', corresponding, by
convention, to the SunOS ethernet device ``le0.''
\let\tt=\COURIERtt
\begin{verbatim}
name=simeth/0;
name=yaap/ip files=yap,yap_clnt,yap_srvr dir=yap,protocols=ip trace=TR_MAJOR_EVENTS;
name=yaap/eth protocols=simeth/0;
name=prt files=prt dir=prt, protocols=yaap/ip,yaap/eth trace=TR_ERRORS;
\end{verbatim}
\let\tt=\CMRtt

\noindent Only the first of multiple instantiations should have {\sanss dir},
{\sanss files}, or {\sanss trace} fields.

The third section contains the names of files with protocol number
tables that are to be loaded during initialization.  It also
contains the names of subsystems and their configuration parameters.
Currently trace variables are the only parameters that can be set
here.  The following illustrates a typical use of the third section:

\let\tt=\COURIERtt
\begin{verbatim}
@;
#
# You can also specify auxiliary protocol tables to be read in at boot
# time.  DEFAULT loads the system protocol table file as listed in etc/site.h
#
prottbl=DEFAULT;
prottbl=./prottbl;

#subsystem tracing for messages and protocol operations
name=msg	trace=TR_GROSS_EVENTS;
name=protocol	trace=TR_MAJOR_EVENTS;
\end{verbatim}
\let\tt=\CMRtt

The graph.comp file is read by an $x$-kernel utility program called
{\em compose} which generates startup code to build the protocol
graph and set up the described configuration.  The protocol graph is
built bottom-up; when a protocol's initialization function is called,
the lower level-protocols have already been initialized.

\subsection{Build Procedure}

A given instance of the \xk{} is built in a working directory.  Each
working directory can support one \xk{} configuration at a time.
Working directories may be organized within the \xk{} source tree
(e.g., as subdirectories of {\sanss xkernel/build}) or outside of
the source tree (e.g., in an \xk{} user's home directory.)  

For the purpose of the following discussion, we assume the user is
configuring a kernel so as to implement and evaluate protocol YAP 
(yet another protocol.) A
user configuring a kernel that contains only existing protocols should
ignore all references to YAP.  The machine architecture in this
example is ``sparc'' (used by the sunos platform); another supported 
architecture is used by the DECstation mach3 platforms (``mips'').

\begin{enumerate}

\item{}
Put {\sanss xkernel/bin/sparc} and {\sanss xkernel/bin} in
your search path. These must be before
{\sanss /bin} and {\sanss /usr/bin}. This will allow use of
the version of {\sanss make} distributed with the $x$-kernel (GNU make
v. 3.57) rather than the standard Unix {\sanss make}.

\item{}
Create a new working directory to hold your kernel.


\item{}
Change to your new working directory.

\item{}
Copy the Makefile
from {\sanss xkernel/build/Template/Makefile.sunos} into your
working directory under the name Makefile:

\begin{quote}
{\sanss cp xkernel/build/Template/Makefile.sunos Makefile}\\
\end{quote}

\item{}
Make the Makefile writable:
\begin{quote}
{\sanss chmod u+w Makefile}\\
\end{quote}

\item{}
Edit the Makefile.
Select an appropriate {\sanss HOWTOCOMPILE}. Generally, you want {\sanss
DEBUG} until you are ready to run serious performance tests, then
change it to {\sanss OPTIMIZE}. (See Section \ref{kdebug}.)

Check to see that the variable XRT in your Makefile is a path to
the root of the xkernel source tree.


\item{}
If you are including a new protocol in your kernel, then create (at least)
a shell for that protocol; e.g., create a shell version of {\sanss yap.c}.
The protocol shell should define an initialization routine; 
e.g., {\sanss yap\_init}.  You will also need to put an entry for your
protocol in one of the protocol tables referenced by your graph.comp
(see section \ref{protnum}.)
If you are configuring a kernel that contains only existing protocols, 
then this step is not necessary.

\item{}
Create
{\sanss graph.comp}, as discussed in section \ref{compose},
to specify the protocol graph for your kernel.  Example graph.comps
are located in the {\sanss xkernel/build/Template} directory.

\smallskip

In steps \ref{en:make_compose} and
\ref{en:make_depend}, you may see one or more compiler warning messages of the
form:

\begin{verbatim}
         make[n]: fopen: Makedep.DEBUGsunos: No such file or directory
\end{verbatim}

These messages are expected and can be ignored.
The dependency files referenced by these messages will be created in
step \ref{en:make_depend}.

The following
two steps must be redone if {\sanss graph.comp} is later modified:
\item{}
\label{en:make_compose}
{\sanss make compose}

\item{}
\label{en:make_depend}
{\sanss make depend} 
Note:  If you are the first person building a kernel at your site
(i.e., if you are building a kernel as part of the installation
procedure), you must run {\sanss make alldepend} here instead of
{\sanss make depend.}  This will create the dependency files in all of
the source directories, not just in your build directory.

The dependency files built for the DEBUG and OPTIMIZE configurations
are different.  The person installing the \xk{} should run {\sanss
make alldepend} for both DEBUG and OPTIMIZE mode for each platform
that will be used.

\item{}
{\sanss make}

\end{enumerate}

\subsection{Debug versus Optimized Mode}

The $x$-kernel uses a trace package to generate debugging information.
To enable the tracing facility, edit the Makefile to set {\sanss
HOWTOCOMPILE} to {\sanss DEBUG}; tracing can be disabled by setting {\sanss
HOWTOCOMPILE} to {\sanss OPTIMIZE}. Then type {\sanss make}.  See also
section \ref{kdebug}.

If you are interested in accurate performance timings, you should set
{\sanss HOWTOCOMPILE} to {\sanss OPTIMIZE} in the Makefile.  This
causes all trace of tracing code to be eliminated form the kernel.





