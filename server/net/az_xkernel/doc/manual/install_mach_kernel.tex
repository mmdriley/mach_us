%
% installing_mach_kernel.tex
%
% $Revision: 1.2 $
% $Date: 1993/02/05 20:58:55 $
%

\section{Installing within a Mach3 Kernel}
\label{installingmachk}

The \xk{} can be compiled as an integral part of the Mach3 kernel.
This facility is provided for researchers developing highly efficient
protocols and for obtaining timing results that are exclusive of
Mach IPC and Unix server overhead.  At the present time, there is no
user interface for accessing the \xk{} services -- the Mach3 kernel
can boot, run an \xk{} timing test and print the results.  A more
usable interface will be provided in the future.

These are instructions for building a version of the \xk{} that resides
in the Mach3 kernel.  You should be familiar with the build instructions
for the ``out of kernel'' version before starting.

In the mach3 src directory, go to the kernel directory and
create a subdirectory ``xkern''; the \xk{} source tree should go here.

Set your path to include the /usr/xkernel/bin and /usr/xkernel/bin/ARCH
directories.

The files distributed (and these directions) have been used with the
MK74 Mach kernel for a DecStation 5000.

\begin{enumerate}
\item cd xkern/include

\item Link all .h files symbolically to the xkern/include directory:

\begin{verbatim}
   ln -s ../etc/site.h
   ln -s prot/*.h .
   ln -s ../pi/include/*.h .
   ln -s ../mach3/include/*.h .
   ln -s ../protocols/*/*.h .
   ln -s ../protocols/util/port_mgr.c
   ln -s ../pi/msg_internal.h
   ln -s ../gen/initRom.c
\end{verbatim}

\item Append the xkern/mach3/machkernel/files.addition file to the
   kernel/conf/files file.

   Append the xkern/mach3/machkernel/files.mips.addition file to the
   kernel/conf/files.mips file.  Comment out the chips/lance.c line from
   the original files.mips.

\item Add the lines from xkern/mach3/machkernel/MASTER.local, MASTER.mips
   and Makefile.mips to the corresponding files in the mach kernel conf
   directory.  The variable XKRT in MASTER.local may have to be edited
   to reflect the path to the \xk{} source directory in your local
   environment.

   Append the variable \${XLIB} to the definition of CC\_FLAGS\_NORMAL in
   mk/kernel/conf/Makefile.template.  The definition should read:
\begin{verbatim}
	CC_FLAGS_NORMAL = -c ${CDEBUGFLAGS} ${CPP_ENV} ${VOLATILE} ${XLIB}
\end{verbatim}


\item Two files must be copied to mach3 kernel area:
\begin{verbatim}
	xkern/mach3/machkernel/device_init.c   replaces   device/device_init.c
	xkern/mach3/drivers/xklance/xklance.c  copied as  chips/xklance.c
\end{verbatim}

\item Create a graph.comp in a build directory as for an out-of-kernel
   version, but with the in-kernel driver (xklance) instead of the
   out-of-kernel driver (ethdrv.)  See the xklance man page for
   configuration options.

   run {\tt compose -f < graph.comp}. Copy the following files to the 
   xkern/gen directory:
\begin{verbatim}
	traceLevels.c
	protocols.c
	protocols.h
	protTbl.c
\end{verbatim}

   (default versions of these files are provided, derived from a graph.comp
    file that includes most of our transport protocols, but you must
    regenerate these files each time you change graph.comp)

\item In the xkern/gen directory, run the following command:
\begin{verbatim}
	ptbldump ../etc/prottbl > ptblData.c
\end{verbatim}
   This creates a compilable version of the protocol table.

\item You must also create a ROM file in the gen directory.
    Give the command 
\begin{verbatim}
	awk -f ../bin/genrom.awk < rom > initRom.c
\end{verbatim}
    The rom file may be empty (but the awk command must still be run.)

\item Change your path to *exclude* the \xk{} bin directories.


\item If you need command line arguments, like those used by udptest and tcptest,
   edit the file xkern/mach3/pxk/init.c to simulate command line strings.
   The file has example strings for running the udp test, both for
   the server and client machines.


\item Edit mk/Makefile-config to adjust the PMAX\_CONFIG configuration to
    be 
\begin{verbatim}
	STD+ANY-mips_code+fixpri+xk
\end{verbatim}

    to have x-kernel compiled in ``debug'' mode or 
\begin{verbatim}
	STD+ANY-mips_code+fixpri+xko
\end{verbatim}
    to have the x-kernel compiled in optimized mode.


\item In the mach3 kernel directory (src/mk/kernel), run 'make'.
This will create the Makefile and dependencies in the obj directory and
will compile the Mach kernel with \xk{} additions.


\item If you have problems with kernels crashing because of storage exhaustion,
   you might want to adjust the zone block table in the Mach kernel file
   kalloc.c.  See the \xk{} file xkernel/pi/msg\_internal.h for the
   size of a message block; these account for most of the \xk{} storage use.


\item The machkernel will come up by default with both the Unix server
and the \xk{} receiving and sending network packets (incoming
packets are copied to both the \xk{} and the Unix server.)  For
accurate kernel-kernel timings of \xk{} protocols, boot the
mach\_kernel image (which doesn't start the user task) instead of the
mach.boot image and configure the in-kernel driver (xklance) to not
deliver incoming packets to the user task as described in Appendix A.

\item If you get a load-time error of the form:
\begin{verbatim}
	gp relocation out-of-range errors have occured and bad object file
	produced (corrective action must be taken)
	Best -G num value to compile all -count'ed objects creating 
	mach_kernel.MK74.STD+ANY-mips_code+fixpri+xk.out with is 30
\end{verbatim}

consider changing the value of the GPSIZE flag in
kernel/conf/MASTER.mips from 32 to 0 and rebuilding from scratch.
This seems to allow the kernel to build, although the user process
doesn't seem to come up properly under this modification.  This
error can also be avoided by not linking in as many \xk{}
protocols.

\end{enumerate}
