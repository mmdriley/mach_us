\documentstyle[11pt]{cmu-art}

\newcommand{\comment}[1]{}
%\newcommand{\MORE}{(More-Text-Here)}
\newcommand{\MORE}{}
\newcommand{\CHECK}{(Check-This-Info)}

\topmargin 0in
\textheight 9in
\pagestyle{empty}
\makeatletter
\def\@oddfoot{\hfil\em DRAFT-\arabic{page}\hfil}
\let\@evenfoot=\@oddfoot

\def\@listi{
\leftmargin\leftmargini 
\topsep 4pt plus 2pt minus 2pt
\parsep 3pt plus 1pt minus 1pt
\itemsep \parsep}

\def\section{\@startsection {section}{1}{\z@}{-1.6ex plus -.2ex minus 
 -.2ex}{.01ex plus .1ex}{\large\sf\bf}}
\def\subsection{\@startsection{subsection}{2}{\z@}{-1.4ex plus -.2ex minus 
 -.2ex}{.01ex plus .1ex}{\normalsize\sf\bf}}


\makeatother

%\input draft
%\special{' @draft}
\special{! /@scaleunit 100 def }

\title{Mach US: Current Status}
\author{J. Mark Stevenson \\
\\
School of Computer Science \\
Carnegie Mellon University \\
Pittsburgh, PA 15213 \\
jms@cs.cmu.edu \\
}
\date{October, 1994}

\begin{document}

\maketitle
\section{Code Status}
The system can now be built using the same tools
and compiler as the rest of the Mach3.0 system (see ``Building Mach 3.0''
in \verb|.../public/doc/unpublished/mach3_build.{doc,ps}.|)  Further build info
can be found in ``Installing and Running Mach-US'' in the file\\
\verb|.../public/doc/unpublished/us-install.{doc,ps}|


\section{Current Usability}
\begin{itemize}
\item{Self Hosting}: Mach-US has been used to build itself.

\item{Test Software Development}: It is used as the default platform for
development of test software used to test Mach-US functionality.

\item{Andrew Benchmark}:  Runs.

\item{Parallel Compile Test}:  There is a version of the standard Mach compile-test
(which just compiles a number of small programs) that runs many compile tests
at the same time.  Mach-US runs this test with heavy loads on uni-processor
and shared memory multi-processor boxes.  

\item{Day to day tools}: Most of the common Unix utilities we know and love
(csh, vi, gnu-emacs, find, ...) are working without problems.

\item{FTP(d)/Telnet(d)/Inetd}: Are used regularly.
\end{itemize}
\section{Current Reliability}
Positive:
\begin{itemize}
\item{Runs and works}: One can log into this Unix system and 
do useful work today.

\item{Typing at a cshell}: One has full tty and job control,
pipes and signals, file access and access control.
It is very much a real Unix.

\item{Stays up indefinitely}:
There is no noticeable system rot.  It does not crash
in the night, or during average use.

Handles heavy computation stress:  The system has been stressed by various
tests, self-hosting, and general use.  It handles the stress well.
\end{itemize}

Negative:
\begin{itemize}
\item{First time applications uncover bugs}:  When a large new bit of software
runs, it may uncover either a bug or a small missing feature or
esoteric semantic.

\item{Untried code paths}:  There are some code or odd interface features,
that have never been used and hence may not work.

\item{Different error values}: Mach-US may return a different error or errno
than Mach2.5 for the same error.
\end{itemize}

\section{Speed}
In general the CMU Mach single-server(Mach-UX) runs 10\verb|%|-20\verb|%|
faster than Mach-US for common usage tests.
Yet in a highly parallel task (parallel compile test) on our 
multi-processor Sequent box,
Mach-US performs slightly better than Mach-UX.
It has also been measured to run much faster than Mach-UX for FTP.

Because of the sophistication of our emulation lib,
most syscalls (as defined by
occurrence during test) do not make any calls to the servers, but instead
are handled in the client process.
By this property,  I believe that Mach-US could be
made to run faster than the Mach-UX system in general.
Furthermore, with the exception of remote method invocation,
we have yet to find the time to
meter it heavily to discover what slowdowns exist.
Simple changes alone, driven by such metering, should make the system as fast
or faster than Mach-UX.

There are other obvious places for speedup:  bundling common server call
sequences, shared libraries for server text,
optimizing forking and signaling, and optimizing out some
debugging mechanisms.

\section{Warnings}
\begin{itemize}
\item
The TTY server is slow/clumsy, most notably the ``VUS'' version for
running beside the single server.  This appears to be because of the
use of a dull blade to slice it from the earlier BSD world.  The worst
effect of this is that keyboard interrupts are slow to take effect.
Quite a bit of data can find its way through to the screen before
your ``\verb|^C|'' takes effect.  DON'T BOUNCE ON IT.  Because of bugs in the
system, such repeated pounding of interrupts can cause problems.

Also remember that you can always use \verb|mkill|
to waste any emulated process
safely.  Unlike the Mach-UX single server, Mach-US detects task death,
not just a combination of \verb|kill| and \verb|exit|.

\item
The network server has some mixed bugs on apparent connection lifetime.
In general this is not a big problem, but in some contexts, repeated
connections and disconnections can have problems with naming clashes.
This problem should be fixed in the next release.
\end{itemize}

\section{Missing features}
As of 10/94 there were a few features missing.  The features are missing,
not because of any structural problems, but primarily because of a lack of
developer time.  These are as follows.
\begin{itemize}
\item{Setuid exec}:
For fully correct setuid exec, it is necessary to pickle the state of
the emulation lib that can survive an exec, have the task\_master create
a new task with the right ids, filter the pickled state and inject it into
the new process.  All of this has been postponed.  An incomplete answer
that will be implemented soon, is to just have the task\_master bless a process
with the new ids.

At the current time, the only way to run something as root is to run it from a
shell that is running as root.  The initial process is started as root.

\item{Device Server}:
There is currently no way to get at devices directly.  The tty\_server handles
ttys, the ufs servers handle disk partitions, and right now, that is it.
Most notably, there is no access to a screen device for X and the such.
The device server is expected soon.  It has yet to be done because it is so
simple.  The server merely needs to support "opens" for device objects,
an intelligent proxy will be able to translate all other requests directly into
kernel calls for the device in question.

\item{Mixed Admin Stuff}:
It was not considered an early priority to get all of the system administrator
style features to work in the classic Unix fashion.
Many administrative features do not work, or do nothing.  mount/unmount
do not work because this function is done differently.
(see the ``fsadmin'' explanation in the
document entitled ``Installing and Running Mach-US'').
Sync is not currently connected and
will return successfully without doing anything.
(wait a bit and the UFS servers
will sync).  The utilities for network configuration and administration
generally do nothing at all because the network subsystem is totally unrelated
to that of BSD,  Again see the ``Installing and Running Mach-US'' document.

\item{Xwindows}: No X applications have been run to date, and the there
is currently no way for a default Xserver to access the screen device.
\end{itemize}

\subsection{Compliance Testing}
We have run some 4.3BSD compliance tests from PERENNIAL Inc.
Many of the bugs uncovered were fixed.  Some bugs were problems with
the tests themselves or were concerned with features that
we did not implement.  There are several bugs remaining:  mixed problems with
esoteric semantics of sockets, esoteric signal semantics, and the tsymlinks
to ttys in the ``/dev'' area don't work for the ``tty'' program.

\section{Futures}
There are no new features expected from CMU for this system in the future.
We are now in distribution/support/maintainence mode on this system.
New and Old
system users will be supported, bugfixes will be accepted, and some
additional bug fixes may be made when they are truly needed.
\comment{
The following things are expected to be available in a release in the
near future:
\begin{itemize}
\item{Mixed Bug Fixes}
\item{Ps/Ms Style Exec Strings}
\item{``Free'' subset made available via anon FTP}
\end{itemize}

Some other possibilities in the medium term are:
\begin{itemize}
\item{Mixed Bug Fixes}
\item{Setuid Exec}
\item{X Window Clients and Server}
\end{itemize}

The long term prospects for this system are not totally clear.   Here at CMU,
multiple changes in facility, staff, and students have restricted development.
}
\end{document}
